\section{Agrupamiento}

Los caracteres \texline|{| y \texline|}| tienen una función especial en \TeX. Estos permiten crear ``bloques'' donde los comandos tiene un efecto local. También puede ser usado para romper ligaturas, para evitar que los espacios consecutivos se combinen o que se eliminen después de un comando.

Sin embargo el uso principal en \TeX\ es para delimitar el argumento de algunos comandos. Por ejemplo, el siguiente comando creará una línea centrada con la frase ``This information should be centered''.
\begin{texcode}
  \centerline{This information should be centered.}
\end{texcode}

En el argumento de un comando pueden haber otros grupos, como en el siguiente ejemplo.
\begin{texcode}
  \centerline{This information should be {\it centered}.}
\end{texcode}

\begin{notebox}
  Notar que el primer grupo sirve como argumento, mientras que el segundo funciona como un bloque. La diferencia es que el primer comando \emph{requiere} un argumento, pero el segundo \emph{cambia} uno o varios comando.
\end{notebox}

Una forma de aplicar cambios globales adentro de un grupo es mediante el comando \texcs\global. Por ejemplo, el siguiente ejemplo incrementará el registro \texline{\count0} en una unidad y el cambio se preservará aun despúes de que se sale del grupo.
\begin{texcode}
  {\global\advance\count0 by 1}
\end{texcode}

Algo que hay que notar es que \texcs\global\ se salta todos los grupos, no solo el primer en el que está contenido.

Otra forma de crear grupos, además de los caracteres \texline|{| y \texline|}| es mediante los comandos \texcs\begingroup\ y \texcs\endgroup. Su principal utilidad es que estos comandos pueden estar dentro del texto de reemplazo en el comando \texcs\def, a diferencia de los corchetes.

Notar que estos dos métodos para crear grupos no se pueden mezclar, por ejemplo el siguiente código no es válido:
\begin{texcode}
  { \begingroup } \endgroup % ERROR!!!
\end{texcode}
