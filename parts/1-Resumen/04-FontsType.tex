\section{Fuentes tipográficas}

\TeX\ cuenta con múltiples tipos de fuentes y comandos para cambiar estos. 
Cada fuente cuenta con su propio comando para cambiarlo, algunas de estos comandos se muestran en el siguiente ejemplo.

\begin{texexample}
  \rm roman typeface.
  \sl slanted romand typeface.
  \it italic style.
  \tt typewriter-like face.
  \bf boldface style.
\end{texexample}

Estos comandos cambian la fuente para el resto del documento, sim embargo se puede limitar a una parte mediante el uso de las llaves \texline|{| y \texline|}|.

\begin{texexample}
  to be {\bf bold} or {\sl emphasize} something
\end{texexample}

En el caso de las fuentes slanted e italic, por la forma de las letras, es posible que la última letra abarque parte del siguiente espacio. Para corregir esto se puede usar el comando \texcs\/ el cual agrega un espacio especial llamado \emph{correción itálica}.

\begin{texexample}
  {\it italic\/} and {\sl slanted\/} words.
\end{texexample}

La regla de oro es usar \texcs\/ justo cuando se cambia de italic o slanted a roman, excepto cuando el siguiente carácter se un punto o coma.

\begin{texexample}
  {\it italic\/} for {\it emphasis}.
\end{texexample}

La puntuación no debe ser incluida en el cambio de fuente. En el caso de los signos de dos puntos `:' o el punto y coma `;' si se recomienda agregar la corrección itálica \texline|{\it word\/};|.

Todas las letras en todas las fuentes tienen una correción itálica. En fuentes no inclinadas usualmente es de cero, sin embargo existe excepciones. Una de estas excepciones son las negritas, dado que si queremos colocar la letra f negrita entre comillas, se debe hacer como \texline|`{\bf f\/}'|.

Otra fuente presente en \TeX\ es la primitiva \texcs\nullfont, este define una fuente vacía y está siempre presente en el caso que no se haya especificado otra.

\TeX\ también puede manejar fuentes de distintos tamaños. Por defecto se usa una fuente de 10pt llamada \texcs\tenrm, pero se puede usar otros tamaños usando los comandos \texcs\ninerm, \texcs\eightrm, \texcs\sevenrm, \texcs\sixrm\ y \texcs\fiverm, los cuales cambia a la fuente roman de tamaños 9pt a 5pt, respectivamente. Existen comandos similares para otros estilos, como \texcs\tensl, \texcs\ninesl, etc. En el caso que en una misma fila se use fuentes de distinto tamaño, estas se alinearan mediante su linea base (baseline).

{
\font\tenrm=cmr10  \font\ninerm=cmr9  \font\eightrm=cmr8
\font\sevenrm=cmr7  \font\sixrm=cmr6  \font\fiverm=cmr5
\begin{texexample}
  \tenrm smaller \ninerm and smaller
  \eightrm and smaller \sevenrm and smaller
  \sixrm and smaller \fiverm and smaller
\end{texexample}
}

\begin{notebox}
  En realidad, en plain \TeX\ solo están definidos por defecto los comandos para los tamaños 10pt, 7pt y 5pt, el resto de tamaños únicamente están precargados, por lo que para usarse se tienen que definir explícitamente mediante el comando \texcs\font.
\end{notebox}


En plain \TeX\ no hay diferencia notable entre \texcs\rm\ y \texcs\tenrm, sin embargo, \texcs\rm\ puede ser reescrito para que cambie la forma más no el tamaño de la fuente.

En \TeX\ se puede cargar cualquier fuente siempre que existan los archivos y se tenga su nombre clave. Si estos requisitos se cumple, se puede usar el comando \texcs\font\ para cargar y asignar un comando a esa fuente. La sintaxis es la siguiente

\begin{texcode}<escapeinside=||>
  \font|\inputparameter{control sequence}|=|\inputparameter{font name}|
\end{texcode}

Por ejemplo el comando \texcs\ninerm\ está definida de la siguiente manera.
\begin{texcode}
  \font\ninerm=cmr9
\end{texcode}

En el ejemplo anterior, \texline{cmr9} es el nombre clave para la fuente ``Computer Modern Roman de 9pts''.

\begin{notebox}
  Las fuentes que se pueden usar en plain \TeX\ son de un formato especial. Se crean usando su programa hermano METAFONT.

  Si se requiere usar fuentes modernas, se debe hacer uso de otros ``engines'' de \TeX\ como \XeTeX\ o \LuaTeX.
\end{notebox}

El comando \texcs\font\ también permite cambiar el tamaño de la fuente cargada mediante la palabra clave \texline{at}. La sintaxis es la siguiente.
\begin{texcode}<escapeinside=||>
  \font|\inputparameter{control sequence}|=|\inputparameter{font name}|at|\inputparameter{size}|
\end{texcode}

Por ejemplo para cargar la fuente ``Computer Modern Roman de 5pt'' y estirarla para que su tamaño final sea de 10pt se puede usar el siguiente comando.
\begin{texcode}
  \font\magnifiedfiverm=cmr5 at 10pt
\end{texcode}

Otra forma de modificar el tamaño de una fuente es mediante la palabra clave \texline{scaled} que sirve para escalar una fuente. La diferencia con \texline{at} es que en vez de \inputparameter{size} se deben especificar un factor de escalado que representa la escala multiplicada por 1000.

Por ejemplo, para cargar la fuente ``Computer Modern Roman de 5pt'' al doble de su tamaño se puede usar el siguiente comando.
\begin{texcode}
  \font\magnifiedfiverm=cmr5 scaled 2000
\end{texcode}

Por motivos históricos, \TeX\ proporciona algunos comandos que sirven como abreviaciones para escalas fijas:
\begin{itemize}
  \item \texcs\magstephalf\ para una escala de $\sqrt{1.2}$.

  \item \texcs\magstep\inputparameter{$n = 1,\ldots,5$} para una escala de $1.2^n$
\end{itemize}

Por ejemplo para cargar la fuente ``Computer Modern Roman de 10pt'' escalado en un factor de $1.2^2 = 1.44$ se usa el siguiente comando.
\begin{texcode}
  \font\bigtenrm=cmr10 scaled\magstep2
\end{texcode}
