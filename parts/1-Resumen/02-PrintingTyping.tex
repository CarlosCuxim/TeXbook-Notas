\section{Impresión de libros versus escritura ordinaria}

Las computadoras tiene un conjunto limitado de caracteres (como
ASCII), sin embargo, en la impresión existe mucho más; \TeX\
permite solventar esas diferencias.

Por ejemplo, las computadoras solo tiene 3 tipos de comillas que
son \codeline{`}, \codeline{'} y \codeline{"}, mientras que en
los libros hay muchos mas. Para escribir las comillas en \TeX\ se
usa las siguiente construcción:
\begin{texexample}
  ``I understand''
\end{texexample}
Otro ejemplo son los símbolos de tipo guión. Existen 4 tipos y su uso varía dependiendo de las reglas ortográficas.
\begin{texexample}
  El guión se usa para palabras compuestas como físico-químico.
  El en-dash se usa rangos de números como 10--20.
  El em-dash se usa para énfasis de oraciones o diálogos ---como este---.
  Finalmente tenemos el signo de menos $a - b$.
\end{texexample}

Otra característica importante de \TeX\ son las ligaturas que son símbolos especiales asociados a ciertas combinaciones de letras o símbolos. Por ejemplo tenemos `ff', `fi', `fl', `ffi' y `ffl'. Estos están codificados en la fuente y no hay necesidad de hacer algo especial.

Otra característica de la fuente, llamada \emph{kerning}, es que el espaciado entre letras no es fijo, sino que se adapta para que sea lo más legible posible, como por ejemplo la combinación `AV'

Los símbolos que se usan en el manual y para escribir código son
los siguientes:
\begin{codeblock}[text]
  ABCDEFGHIJKLMNOPQRSTUVWXYZ
  abcdefghijklmnopqrstuvwxyz
  0123456789"#$%&@*+-=,.:;?!
  ()<>[]{}`'\|/_^~
\end{codeblock}
En el caso del espacio en blanco, lo denotaremos como \inputkey{space} o usando el símbolo \textvisiblespace.

En el caso que no se tengan algunos símbolos, es posible reemplazarlo por comandos. Por ejemplo para las comillas podemos usar los comando \texcs\lq\ y \texcs\rq.
\begin{texexample}
  \lq\lq Texto entre comillas \rq\rq
\end{texexample}

Por las mismas características de \TeX, algunas construcciones necesitan consideraciones extra. Por ejemplo, para colocar una comilla doble y una simple al mismo tiempo es necesario usar el comando \texcs\thinspace.
\begin{texexample}
  ``\thinspace` Texto entre dobles comillas'\thinspace''
\end{texexample}
