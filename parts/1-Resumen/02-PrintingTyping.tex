\section{Impresión de libros versus escritura ordinaria}

Las computadoras tiene un conjunto limitado de caracteres (como
ASCII), sin embargo, en la impresión existe mucho más; \TeX\
permite solventar esas diferencias.

Por ejemplo las computadoras solo tiene 3 tipos de comillas que
son \codeline{`}, \codeline{'} y \codeline{"}, mientras que en
los libros hay muchos mas. Para escribir las comillas en \TeX\ se
usan las siguiente construcción:
\begin{texexample}
  ``I understand''
\end{texexample}

Los símbolos que se usan en el manual y para escribir código son
los siguientes:
\begin{codeblock}
  ABCDEFGHIJKLMNOPQRSTUVWXYZ
  abcdefghijklmnopqrstuvwxyz
  0123456789"#$%&@*+-=,.:;?!
  ()<>[]{}`'\|/_^~
\end{codeblock}
En el caso del espacio en blanco, lo denotaremos como \keybutton{space} o usando el símbolo \textvisiblespace.

Otro ejemplo son los símbolos de tipo guión. El guión simple - se escribe como \codeline{-}, el en-dash -- como \codeline{--}, el em-dash como \codeline{---} y el signo de menos $-$ como \codeline{$-$}.
