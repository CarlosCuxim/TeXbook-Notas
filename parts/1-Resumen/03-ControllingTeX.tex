\section{Controlando \TeX}

Para usar \TeX\ haremos uso de las \emph{secuencias de control} ( o simplemente comandos). Estas son instrucciones que se podrán usar mediante un carácter especial llamado \emph{carácter de escape}, el cual usualmente será \codeline{\ }.

La sintaxis de una secuencia de control usualmente será \inputparameter{esc char}\inputparameter{cs name}, donde \inputparameter{esc char} es el carácter de escape y \inputparameter{cs name} es una sucesión de letras o un carácter no letra.

Por ejemplo consideremos el comando \texcs\input\ el cual permite leer un archivo. De este modo si quisiéramos cargar un archivo llamado \codeline{MS.tex} solo tendríamos que escribir lo siguiente
\begin{texcode}
  \input MS
\end{texcode}

Otro ejemplo de secuencias de control son los comandos \codeline{\'} y \codeline{\"} que permiten agregar acentos y a las letras.
\begin{texexample}
  George P\'olya y Gabor Szeg\"o
\end{texexample}

Cuando el comando es del primer tipo, es decir cuando \inputparameter{cs name} es una sucesión de letras, entonces será llamada una \emph{palabra de control}. El comando empieza con la primera letra y termina justo antes del primer carácter que no sea una letra. Los espacios después de una palabra de control serán ignorados.

Sin embargo, cuando el comando es del segundo tipo, cuando \inputparameter{cs name} es un carácter que no es letra, será llamado un \emph{símbolo de control}. Los símbolos de control siempre tienen una carácter de longitud (sin contar al carácter de escape). Los espacios después de un símbolo de control nunca serán ignorados.

Si se desea un espacio después de una palabra de control se deberá usar el comando \texline<escapeinside=||, showspaces>{\| |}. No importa la cantidad de espacios después del comando, ya que espacios consecutivos son tratados como un solo espacio.

Por ejemplo, el comando \texcs\TeX\ es usado para imprimir el logo de \TeX, para usarlo debemos hacer lo siguiente
\begin{texexample}
  \TeX ignore spaces after control words.
  \TeX\ ignore spaces after control words.
\end{texexample}

Si se desean más de un espacio consecutivo se debe usar \texline<escapeinside=||, showspaces>{\| |} múltiples veces. Por ejemplo, espacios consecutivos se deberá escribir como \texline<showspaces, escapeinside=||>{\ \ \| |}.

Oros caracteres de tipo espacio como \inputkey{return} o \inputkey{tab} también pueden generar caracteres de escape. Estos comandos, \texline{\ }\inputkey{return} y \texline{\ }\inputkey{tab}, usualmente tendrán el mismo significado que \texline<escapeinside=||, showspaces>{\| |}.

Otros ejemplos de comandos son los usados para los símbolos en formulas matemáticas, sin embargo, para usarlos estos deben estar entre uno o dos caracteres \mintinline{latex}{$}.

\begin{texexample}
  $\pi$, $\Pi$, $\aleph$, $\infty$, $\le$, $\ge$,
  $\ne$, $\oplus$, $\otimes$.
\end{texexample}

Todos los comandos son \emph{case sensitive}, es decir, que no hay relación entre comandos definido por la misma palabra pero cambiando algunas mayúsculas o minúsculas.

\TeX\ tiene internamente cerca de 900 comandos listos para ser usados. Pero existen formas de definir mas. De estos comandos, cerca de 300 son llamada \emph{primitivas}, dado que son comandos internos del propio programa y no se descomponen en comandos más simples.

Se pueden distinguir las primitivas y los comandos compuestos usando el comando \texcs\show\ seguido del comando. Por ejemplo, \texline{\show\thinspace} hace que en la consola y el log se muestre el siguiente texto:
\begin{texcode}
  > \thinspace=macro
  ->\kern .16667em .
\end{texcode}

Llamaremos como ``plain \TeX'' al conjunto de casi 600 comandos ya integrados dentro del programa más los casi 300 comandos primitivos.

