% !TEX root = ./main.tex
\documentclass[theme=mocha, pagecolor, pagesize=a5paper, stretchmode]{qx-files/qx-notes}

\qxsetmainlanguage{latex}

\usepackage[spanish, mexico]{babel}

\usepackage{amsmath, amssymb}

\title{TeXbook Notas}
\author{Qx}
\date{\today}


\begin{document}
  \maketitle

  \begin{abstract}
    En estas notas colocaré un resumen del TeXbook. El objetivo es dar un desglose de todos los comandos y sus funciones en español, así como dar más información relacionada con las pruebas que he hecho.
  \end{abstract}

  \part{Resumen del libro}

  \section{El nombre del juego}

El nombre de \TeX\ viene de la palabra griega $\tau\epsilon\chi$ que es la raíz griega de palabras como ``tecnologia'' y significa arte. Por ende se debería pronunciar como ``tej'' en vez de ``teks''.

Este enfoque es para recordar que \TeX\ es un sistema para hacer documentos tipográficos de alta calidad.

La E en \TeX\ está bajada para mostrar las altas capacidades del programa. Así, para escribirlo en texto plano se debe escribir como \codeline{TeX}.


  \section{Impresión de libros versus escritura ordinaria}

Las computadoras tiene un conjunto limitado de caracteres (como
ASCII), sin embargo, en la impresión existe mucho más; \TeX\
permite solventar esas diferencias.

Por ejemplo, las computadoras solo tiene 3 tipos de comillas que
son \codeline{`}, \codeline{'} y \codeline{"}, mientras que en
los libros hay muchos mas. Para escribir las comillas en \TeX\ se
usa las siguiente construcción:
\begin{texexample}
  ``I understand''
\end{texexample}
Otro ejemplo son los símbolos de tipo guión. Existen 4 tipos y su uso varía dependiendo de las reglas ortográficas.
\begin{texexample}
  El guión se usa para palabras compuestas como físico-químico.
  El en-dash se usa rangos de números como 10--20.
  El em-dash se usa para énfasis de oraciones o diálogos ---como este---.
  Finalmente tenemos el signo de menos $a - b$.
\end{texexample}

Otra característica importante de \TeX\ son las ligaturas que son símbolos especiales asociados a ciertas combinaciones de letras o símbolos. Por ejemplo tenemos `ff', `fi', `fl', `ffi' y `ffl'. Estos están codificados en la fuente y no hay necesidad de hacer algo especial.

Otra característica de la fuente, llamada \emph{kerning}, es que el espaciado entre letras no es fijo, sino que se adapta para que sea lo más legible posible, como por ejemplo la combinación `AV'

Los símbolos que se usan en el manual y para escribir código son
los siguientes:
\begin{codeblock}[text]
  ABCDEFGHIJKLMNOPQRSTUVWXYZ
  abcdefghijklmnopqrstuvwxyz
  0123456789"#$%&@*+-=,.:;?!
  ()<>[]{}`'\|/_^~
\end{codeblock}
En el caso del espacio en blanco, lo denotaremos como \inputkey{space} o usando el símbolo \textvisiblespace.

En el caso que no se tengan algunos símbolos, es posible reemplazarlo por comandos. Por ejemplo para las comillas podemos usar los comando \texcs\lq\ y \texcs\rq.
\begin{texexample}
  \lq\lq Texto entre comillas \rq\rq
\end{texexample}

Por las mismas características de \TeX, algunas construcciones necesitan consideraciones extra. Por ejemplo, para colocar una comilla doble y una simple al mismo tiempo es necesario usar el comando \texcs\thinspace.
\begin{texexample}
  ``\thinspace` Texto entre dobles comillas'\thinspace''
\end{texexample}


  \section{Controlando \TeX}

Para usar \TeX\ haremos uso de las \emph{secuencias de control} ( o simplemente comandos). Estas son instrucciones que se podrán usar mediante un carácter especial llamado \emph{carácter de escape}, el cual usualmente será \texline{\ }.

La sintaxis de una secuencia de control usualmente será \inputparameter{esc char}\inputparameter{cs name}, donde \inputparameter{esc char} es el carácter de escape y \inputparameter{cs name} es una sucesión de letras o un carácter no letra.

Por ejemplo consideremos el comando \texcs\input\ el cual permite leer un archivo. De este modo si quisiéramos cargar un archivo llamado \codeline{MS.tex} solo tendríamos que escribir lo siguiente
\begin{texcode}
  \input MS
\end{texcode}

Otro ejemplo de secuencias de control son los comandos \texline{\'} y \texline{\"} que permiten agregar acentos y a las letras.
\begin{texexample}
  George P\'olya y Gabor Szeg\"o
\end{texexample}

Cuando el comando es del primer tipo, es decir cuando \inputparameter{cs name} es una sucesión de letras, entonces será llamada una \emph{palabra de control}. El comando empieza con la primera letra y termina justo antes del primer carácter que no sea una letra. Los espacios después de una palabra de control serán ignorados.

Sin embargo, cuando el comando es del segundo tipo, cuando \inputparameter{cs name} es un carácter que no es letra, será llamado un \emph{símbolo de control}. Los símbolos de control siempre tienen una carácter de longitud (sin contar al carácter de escape). Los espacios después de un símbolo de control nunca serán ignorados.

Si se desea un espacio después de una palabra de control se deberá usar el comando \texline<escapeinside=||, showspaces>{\| |}. No importa la cantidad de espacios después del comando, ya que espacios consecutivos son tratados como un solo espacio.

Por ejemplo, el comando \texcs\TeX\ es usado para imprimir el logo de \TeX, para usarlo debemos hacer lo siguiente
\begin{texexample}
  \TeX ignore spaces after control words.
  \TeX\ ignore spaces after control words.
\end{texexample}

Si se desean más de un espacio consecutivo se debe usar \texline<escapeinside=||, showspaces>{\| |} múltiples veces. Por ejemplo, espacios consecutivos se deberá escribir como \texline<showspaces, escapeinside=||>{\ \ \| |}.

Oros caracteres de tipo espacio como \inputkey{return} o \inputkey{tab} también pueden generar caracteres de escape. Estos comandos, \texline{\ }\inputkey{return} y \texline{\ }\inputkey{tab}, usualmente tendrán el mismo significado que \texline<escapeinside=||, showspaces>{\| |}.

Otros ejemplos de comandos son los usados para los símbolos en formulas matemáticas, sin embargo, para usarlos estos deben estar entre uno o dos caracteres \mintinline{latex}{$}.

\begin{texexample}
  $\pi$, $\Pi$, $\aleph$, $\infty$, $\le$, $\ge$,
  $\ne$, $\oplus$, $\otimes$.
\end{texexample}

Todos los comandos son \emph{case sensitive}, es decir, que no hay relación entre comandos definido por la misma palabra pero cambiando algunas mayúsculas o minúsculas.

\TeX\ tiene internamente cerca de 900 comandos listos para ser usados. Pero existen formas de definir mas. De estos comandos, cerca de 300 son llamada \emph{primitivas}, dado que son comandos internos del propio programa y no se descomponen en comandos más simples.

Se pueden distinguir las primitivas y los comandos compuestos usando el comando \texcs\show\ seguido del comando. Por ejemplo, \texline{\show\thinspace} hace que en la consola y el log se muestre el siguiente texto:
\begin{texcode}
  > \thinspace=macro
  ->\kern .16667em .
\end{texcode}

Llamaremos como ``plain \TeX'' al conjunto de casi 600 comandos ya integrados dentro del programa más los casi 300 comandos primitivos.




\end{document}
